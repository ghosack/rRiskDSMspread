\documentclass[12pt]{paper}

\usepackage{latexsym,amssymb,amsfonts,amsmath}
\usepackage[pdftex]{graphicx}
% or use the graphicx package for more complicated commands
\usepackage{graphicx, graphics}
\usepackage[round]{natbib}
\usepackage{blindtext,tikz}
\usetikzlibrary{calc}
\usepackage{datetime}
\usepackage[obeyspaces]{url}
\usepackage{amsthm, amsmath, amssymb}
\usepackage{geometry}
\geometry{margin = 60pt}
\usepackage{setspace}
\doublespacing
%% The lineno packages adds line numbers. Start line numbering with




%%% Put your definitions there:
\title{List of changes: addressing the comments}

\begin{document}

\maketitle

%\linenumbers
We would like to thank again the reviewers for this last round of comments. We addressed this time the remarks on reviewer-by-reviewer basis. \\

\section*{Reviewer A:}

\emph{
\footnotesize
\begin{itemize}
\item  \\
{\color{red} - . }
\item I believe that one of the reviewers asked for more information regarding the model by Szpiro et al. (2010) on which this model builds on, but the authors have not provided any background information. I think this should be included. In particular, it would be worthwhile to understand why is the paper by Szpiro et al. (2010) not appropriate for this data. Why is there a need for a mixture model?\\
\end{itemize}
}
~\\
~\\

\newpage

\section*{Reviewer B:}
\emph{
\footnotesize
\begin{itemize}
\item  \\
{\color{red} }
\end{itemize} 
}
~\\
~\\
\end{document}

%%
%% End of file `elsarticle-template-1a-num.tex'.
