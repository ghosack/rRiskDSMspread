\documentclass[12pt]{paper}

\usepackage{latexsym,amssymb,amsfonts,amsmath}
\usepackage[pdftex]{graphicx}
% or use the graphicx package for more complicated commands
\usepackage{graphicx, graphics}
\usepackage[round]{natbib}
\usepackage{blindtext,tikz}
\usetikzlibrary{calc}
\usepackage{datetime}
\usepackage[obeyspaces]{url}
\usepackage{amsthm, amsmath, amssymb}
\usepackage{geometry}
\geometry{margin = 60pt}
\usepackage{setspace}
\doublespacing
%% The lineno packages adds line numbers. Start line numbering with




%%% Put your definitions there:
\title{List of changes: addressing the comments}

\begin{document}

\maketitle

%\linenumbers
We would like to thank again the reviewers for this last round of comments. We addressed this time the remarks on reviewer-by-reviewer basis. \\

\section*{Reviewer 1:}

\emph{
\footnotesize
\begin{itemize}
\item  The evaluation of model performance is the biggest shortfall of this paper. The authors note that only 2 out of 70 observations of mosquito captures are outside the $90\%$ credible interval of the posterior predictive expected number of catches (pg. 12 lines 40-45). However, the $90\%$ credible intervals on the predictions are very large. The authors should include additional information on the estimated error in the predicted number of catches (e.g. RMSE etc) and provide a more in-depth discussion of model performance.\\
{\color{red} - Adrien:}
\item  Please add further details clarifying why dispersal of WT mosquitoes might differ from the Sterile GM males that the model is estimating and whether using WT MRR experiments might negatively affect predictions of GM mosquito dispersal.\\
{\color{red} - Adrien and Geoff:}
\item  The methods would benefit from additional discussion of the specific expertise of the experts that were surveyed. Generally, what qualified them to make these predictions/on what basis were the predictions made? Why was this approach preferable to acquiring priors from lab-based studies currently in the literature?\\
{\color{red} - Geoff:}
\item   Were surveys of experts conducted in a standardized way? It would be helpful if you could supply in a supplement the specific questions that each expert was asked and their responses.\\
{\color{red} - Adrien:}
\item  The discussion and/or introduction would benefit if the authors specifically discussed how the results of this study were used to assess risk and how they informed the release of the GM mosquitoes.\\
{\color{red} Additional text has been added to the discussion describing how the results of the study were used to assess risk, and also emphasising that these results are case specific and should not be generalised to other types of genetic control technology (in response to the first comment from reviewer 2). The text also describes how the risk assessment was designed to provide an independent body of evidence that was made publicly available, and therefore available to any relevant national biosafety authorities, in May 2018. We cannot tell, however, if the assessment was actually used by the relevant biosafety authorities in Burkina Faso as we do not know what information they used during their decision making process.}
\item  The supplemental document appended at the end of the manuscript appears to just be the manuscript in pdf format.\\
{\color{red} - Adrien:}
\item  Typo pg. 13, line 61: 'define the bounds of accuracy that are regulators and stakeholders believe are adequate'\\
{\color{red} - Adrien}
\end{itemize}
}
~\\
~\\

\newpage

\section*{Reviewer 2:}
\emph{
\footnotesize
\begin{itemize}
\item  P2L28: The authors introduce genetic containment (male sterility) as a convenient means to meet the recommendations of a "phased-release" strategy, to collect the data required for evaluating risks on GDMMs. Of course the authors are correct here. However, the way this is phrased suggests that genetic containment is a convenient alternative to physical containment. This makes the assumption that the data gathered for, say a DSM male, are equivalent to the risks associated with alternative designs, for example gene drive replacement strategies.\\
{\color{red} Male sterility is not a genetic containment strategy, at least not as the term is defined by Akabri et al 2015, and we specify in the text that male sterility is a reproductive containment strategy. Nonetheless, it was certainly not our intent to imply that genetic containment is a useful strategy for gathering data and evaluating risks of alternative designs involving (for example) population replacement. We  have therefore added additional text to the discussion (in conjunction with a response to reviewer 1) to emphasise that the data provided by this release is case-specific and does not generalise to an alternative genetic constructs.}
\item P2L14: "force effector genes" is inaccurate because suppression gene drives do have to carry an "effector". I would suggest rephrasing to "gene drives to drive engineered alleles or chromosomes through vector…".\\
{\color{red} We have changed this sentence to read ''...that use gene drives to force engineered alleles through vector...''}
\item P2L38: Males of these mosquitoes are reproductively contained. Currently the way it reads suggests that there is full genetic containment. This is not the case here, DSM females are actually fertile - which is how the strain is maintained. Perhaps this needs to be emphasized more strongly also in the discussion because "containment" here only refers to the released population. The population in the lab, an accidental release of females, or contamination of females in the released population - all of these have to be taken into account when arguing that this strain has the benefit of being "contained".\\
{\color{red} We have amended the sentence in the introduction to read ''These sterile male mosquitoes are reproductively contained (but females that carry the sterilising construct are fertile) and could represent the first stage in a three-stage pathway to malaria vector control using a gene drive that results in a male-biased sex ratio''. We have also added additional text in the discussion (in conjunction with a response to reviewer 1) highlighting that the risk assessment, within which this analysis was conducted, also addressed the risks associated with the accidental release of fertile females.}
\item P2L59 vecotr should be vector.\\
{\color{red} Adrien:}
\item P12L50 What does it mean that the authors "held back a proportion of the observation data". The point of this was lost on me - although admittedly with very limited experience with the methods used here that is probably true for many sections. This is not a criticism of the work, just an honest take from a researcher in the field that does not analyse modeling papers in depth on a regular basis.\\
{\color{red} Adrien:}
\end{itemize} 
}
~\\
~\\
\end{document}

%%
%% End of file `elsarticle-template-1a-num.tex'.
